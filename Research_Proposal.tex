\input{preamble}

\begin{document}

\ifpdf
\DeclareGraphicsExtensions{.pdf, .jpg, .tif}
\else
\DeclareGraphicsExtensions{.eps, .jpg}
\fi

%\maketitle


%\begin{abstract}
%\end{abstract}
%-----
\centerline{\textbf{Quantum limited parametric amplifier for millimeter}}
\centerline{\textbf{and submillimeter astronomy}}
%\section*{Introduction} %use the * if you dont want numbering
\section*{Abstract}
\emph{Abstract Text.}
Proposals for the NASA Postdoctoral Research Program should begin with a brief abstract, followed by the body of the proposal, which should contain these components: a) statement of problem, b) background and relevance to previous work, c) general methodology, procedures to be followed, and timeline for completion of each step; d) explanation of new or unusual techniques, e) expected results and their significance and application, and f) literature citations, where appropriate.
\section*{statement of problem}
\section*{background}
scientific case
technical case
background and relevance to previous work
\section*{Methodology}
general methodology, procedures to be followed, and timeline for completion of each step
\section*{Techniques}
explanation of new or unusual techniques
\section*{Objectives}
expected results and their significance and application
publishable results

% \begin{wrapfigure}{r}{0.5\textwidth}
%     \vspace{-20pt}
%     \begin{center}
%         \includegraphics[width=0.45\textwidth]{images/HPGe_MicroCal_Comparison_Plutonium.png}
%     \end{center}
%     \vspace{-20pt}
%     \caption{Gamma ray spectrum of $\sim$ 0.4 g of \textsuperscript{239}Pu + \textsuperscript{240}Pu from 40 keV to 220 keV (inset) as measured by the microcalorimeter array developed by \cite{Bennet2012} (blue) and a leading edge HPGe detector (red). Data is magnified in the the 100 keV region. The proposed detector would produce an spectrum competitive with the microcalorimeter array. Image taken from \cite{Bennet2012}. }
%     \vspace{-10pt}
%   \label{Fig:PuSpectrum}
% \end{wrapfigure} 



% [Table.~\ref{tab:gammadettechnology}]
% \cite{Bennet2012}

% \begin{table}[htbp]
%   \centering
%   \caption{Comparison of technology between the NIST-LANL microcalorimeter and HPGe ionization detector, representing the most sensitive and most used gamma-ray detectors respectively. $C$ is the combined heat capacity of the TES and the absorber, $F_{fano}$ is the fano factor and other parameters are as defined in Table~\ref{tab:detparams}. `Trans' is the percentage of incident 100 keV gammas that make it through the absorber. Figures listed for the proposed detector represent performance objective achievable on paper.}
%     \begin{tabular}{ccccccc}
%     \toprule
%     Technology & Temp & $\Delta E/E$ & $\Delta E(E)$ & Absorber &Trans & Cts/Sec \\
%     \midrule
%     Planar HPGe & $77 K$ & $3 - 5 \cdot 10^{-3}$ & $\sqrt{F_{fano}\cdot\epsilon_{gap}\cdot E}$ & Sn &$23\% ,\; 5 \, mm $ & $40\cdot 10^{3}$ \\
%     NIST-LANL & $0.1 K$ & $4\cdot 10^{-4}$ & $T \sqrt{k_{B}\cdot C}$ & Ge & $ 63\% ,\; 0.38 \, mm$ & $2.5\cdot 10^{3}$ \\
% 	Proposed & $0.1 K$ & $ \sim 10^{-4}$ & $\sqrt{T_{amp}}$\textsuperscript{See Eq.~\ref{hemtnoise}} & Nb & $ 10^{-6}\% ,\; 5 \,mm$ & $40\cdot  10^{3}$ \\
%     \bottomrule
%     \end{tabular}%
%   \label{tab:gammadettechnology}%
% \end{table}%


% \begin{wrapfigure}{l}{0.74\textwidth} %position can be  r l i or o
%     \vspace{-20pt}
%     \begin{center}
%         \includegraphics[width=0.75\textwidth]{images/Dimensioned_Cross_Section_20140802.png}
%     \end{center}
%     \vspace{-20pt}
%     \caption{Magnified cross section of detector concept utilizing vacuum gap microstrip resonators to sense phonons. A patch of low gap superconductor (e.g. Al or TiN) is located beneath each resonator. Phonons impinging on the patch emanating from within the superconductor absorber break Cooper pairs and change its surface impedance which alters the resonance. Likely dimensions are $w$ = 200 $\mu m$; $t$ = 200 $nm$; $d$ = 3 $\mu m$.}
%     \vspace{-20pt}
%   \label{Fig:PhononsImpinge}
% \end{wrapfigure}












 

\begin{thebibliography}{8}
\singlespacing
\footnotesize
\setlength{\itemsep}{0pt}
\bibitem{Bennet2012}  
    Bennet D.A., Horansky R. D. 2012. Rev. Sci. Instru. 83:093113   
\bibitem{Zmuidzinas2012}     
    Zmuidzinas, J. 2012. Ann. Rev. Cond. Mat .Phys. 3:169-214
\bibitem{Vissers2013}
    Vissers M.R., Gao J., Sandberg M, Duff S.M., Wisbey D.S., Irwin K.D, Pappas D.P. 2013. App. Phys. Lett. 102:232603
\bibitem{Day2012}
    Byeong Ho Eom, Day P.K., LeDuc H.G., Zmuidzinas J.,2012. Nature. 8:623-6272013
\bibitem{Mazin2004}
    Mazin  B. 2004. Microwave Kinetic Inductance Detectors. PhD Thesis. Cal. Inst. Tech., Pasadena. 179pp.
\bibitem{Mazin2013} %2024 channel mux
    Mazin B., Meeker S. R., Strader M. J., Szpryt P., et al. 2013. PASP, 125:1348-1361 
\bibitem{Yates2011} %Aluminum
	Yates S. J. C., Baselmans J. J. A., Endo A., et al. 2011. App. Phys. Lett. 99:073505
%\bibitem{Hubmayr2014} % TiN/Ti
%	Hubmayr J., Beall J., Becker B., Cho H.-M., et al. 2014 arXiv:1406.4010v1 [astro-ph.IM]
%\bibitem{McKenney2012}  %non-stoc TiN
%	McKenney C. M., Leduc H. G.,Swenson L. J., Day P. K., et al. 2012. Proc. of SPIE Vol. 8452:84520S-2
\bibitem{Hauser1995}
    Hauser M. 1995. Acoustic waves in crystals: I. Ultrasonic flux Imaging and internal diffraction, II. Imaging Phonons in superconducting niobium. PhD Thesis. U. Illinois at Urbana-Champaign. 131pp.	
\end{thebibliography}


% \paragraph{Outline}
% First we start with a little <  example of the article class, which is an 
% important documentclass. But there would be other documentclasses like 
% book \ref{book}, report \ref{report} and letter \ref{letter} which are 
% described in Section \ref{documentclasses}. Finally, Section 
% \ref{conclusions} gives the conclusions.
% 
% 
% 
% \section{Documentclasses} \label{documentclasses}
% 
% \begin{itemize}
% \item article
% \item book 
% \item report 
% \item letter 
% \end{itemize}
% 
% 
% \begin{enumerate}
% \item article
% \item book 
% \item report 
% \item letter 
% \end{enumerate}
% 
% \begin{description}
% \item[article\label{article}]{Article is \ldots}
% \item[book\label{book}]{The book class \ldots}
% \item[report\label{report}]{Report gives you \ldots}
% \item[letter\label{letter}]{If you want to write a letter.}
% \end{description}







\end{document}

%----
%\section{Introduction}

%\bibliographystyle{plain}
%\bibliography{}
%\end{document}

% Notes:
% Nuclear energy institute: 12.3 % world energy in 2012
% as of May 2014, 30 countries  have 435 power plants, US has 100 w ~1GW capacity
%
% Pu238 for Multi-Mission Radioisotope Thermoelectric Generators (MMRTG, or formerly simply RTGs)
%
% Read more: http://www.universetoday.com/100875/u-s-to-restart-plutonium-production-for-deep-space-exploration/#ixzz38nmVIuRE
